% Bash VI Editing Mode Shortcut Cheat Sheet
%
% by Peteris Krumins (peter@catonmat.net)
% http://www.catonmat.net  -  good coders code, great reuse
%
% 2008.01.08
%

\documentclass{article}

\usepackage[left=1.5cm,top=1cm,right=1.5cm,bottom=1cm,nohead,nofoot]{geometry}

\usepackage[pdftex]{hyperref}
\hypersetup{pdftitle={Bash VI Readline Editing Mode Shortcut Cheat Sheet}}
\hypersetup{pdfauthor={Peteris Krumins (peter@catonmat.net)}}
\hypersetup{pdfkeywords={cheat sheet, cheat sheat, cheet sheet, cheet sheat, terminal, command line, readline, read line, vi, keyboard, shortcut, shortcuts, unix, linux}}
\hypersetup{pdfsubject={http://www.catonmat.net - good coders code, great reuse}}
\hypersetup{colorlinks}

\pagestyle{empty}

% -----------------------------------------------------------------------

\begin{document}

\begin{center}
\Large Readline VI Editing Mode Cheat Sheet \\
\Large Default Keyboard Shortcuts for Bash
\end{center}

\vspace{0.4in}

\renewcommand{\arraystretch}{1.2}
\begin{tabular}{|p{4.5cm}|p{13cm}|}
\hline
\large\textbf{Shortcut} & \large\textbf{Description} \\
\hline
\multicolumn{3}{|l|}{\small\it{Commands for Moving:}} \\
\hline
\textbf{C-a} & beginning-of-line & Move to the beginning of line. \\
\hline
\textbf{C-e} & end-of-line & Move to the end of line. \\
\hline\hline
\large\textbf{Shortcut} & \large\textbf{Description} \\
\hline
\multicolumn{2}{|l|}{\small\it{Switching to Command Mode:}} \\
\hline
\textbf{ESC} & Switch to \textbf{command} mode. \\
\hline
\multicolumn{2}{|l|}{\small\it{Commands for Entering Input mode:}} \\
\hline
\textbf{i} & Insert before cursor. \\
\hline
\textbf{a} & Insert after cursor. \\
\hline
\textbf{I} & Insert at the beginning of line. \\
\hline
\textbf{A} & Insert at the end of line. \\
\hline
\textbf{c{\textless}movement command{\textgreater}} & Change text of a movement command \textbf{{\textless}movement command{\textgreater}} (see below). \\
\hline
\textbf{C} & Change text to the end of line (equivalent to \textbf{c\$}). \\
\hline
\textbf{cc} or \textbf{S} & Change current line (equivalent to \textbf{0c\$}). \\
\hline
\textbf{s} & Delete a single character under the cursor and enter input mode (equivalent to \textbf{c[SPACE]}) . \\
\hline
\textbf{r} & Replaces a single character under the cursor (without leaving command mode). \\
\hline
\textbf{R} & Replaces characters under the cursor. \\
\hline
\textbf{v} & Edit (and execute) the current command in a text editor (an editor in \$VISUAL and \$EDITOR variables or vi). \\
\hline
\multicolumn{2}{|l|}{\small\it{Basic Movement Commands (in \textbf{command} mode):}} \\
\hline
\textbf{l} or \textbf{SPACE} & Move one character right. \\
\hline
\textbf{h} & Move one character left. \\
\hline
\textbf{w} & Move one word or token right. \\
\hline
\textbf{b} & Move one word or token left. \\
\hline
\textbf{W} & Move one non-blank word right. \\
\hline
\textbf{B} & Move one non-blank word left. \\
\hline
\textbf{e} & Move to the end of the current word. \\
\hline
\textbf{E} & Move to the end of the current non-blank word. \\
\hline
\textbf{0} & Move to the beginning of line. \\
\hline
\textbf{\^} & Move to the first non-blank character of line. \\
\hline
\textbf{\$} & Move to the end of line. \\
\hline
\textbf{\%} & Move to the corresponding opening/closing bracket (()'s, []'s and \{\}'s). \\
\hline
\multicolumn{2}{|l|}{\small\it{Character Finding Commands (these are also Movement Commands):}} \\
\hline
\textbf{f\textit{c}} & Move right to the next occurance of \textit{c}. \\
\hline
\textbf{F\textit{c}} & Move left to the previous occurance of \textit{c}. \\
\hline
\textbf{t\textit{c}} & Move right to the next occurance of \textit{c}, then one char backward. \\
\hline
\textbf{T\textit{c}} & Move left to the previous occurance of \textit{c}, then one char forward. \\
\hline
\textbf{;} & Redo the last character finding command. \\
\hline
\textbf{,} & Redo the last character finding command in opposite direction. \\
\hline
\textbf{\textbar} & Move to the \textit{n}-th column (you may specify the argument \textit{n} by typing it on number keys, for example, \textbf{20\textbar}). \\
\hline

\end{tabular}

%---------- examples

%\begin{center}
%\large Examples
%\end{center}
%
%\begin{tabular}{|p{4.5cm}|p{13cm}|}
%
%\hline
%\large\textbf{Example Shortcut} & \large\textbf{Example Description} \\
%\hline
%
%\hline
%
%\end{tabular}

\vfill

\framebox{\parbox{4.5in}{
A cheat sheet by \textbf{Peteris Krumins} (peter@catonmat.net), 2008.

\href{http://www.catonmat.net}{http://www.catonmat.net} - good coders code, great reuse

\vspace{2mm}
\footnotesize{Released under GNU Free Document License.}}}


%---------------------------------------------------------------------------

\newpage

\mbox{}
\begin{tabular}{|p{4.5cm}|p{13cm}|}
\hline
\multicolumn{2}{|l|}{\small\it{Deletion Commands:}} \\
\hline
\textbf{x} & Delete a single character under the cursor. \\
\hline
\textbf{X} & Delete a character before the cursor. \\
\hline
\textbf{d{\textless}movement command{\textgreater}} & Delete text of a movement command \textbf{{\textless}movement command{\textgreater}} (see above). \\
\hline
\textbf{D} & Delete to the end of the line (equivalent to \textbf{d\$}). \\
\hline
\textbf{dd} & Delete current line (equivalent to \textbf{0d\$}). \\
\hline
\textbf{CTRL-w} & Delete the previous word. \\
\hline
\textbf{CTRL-u} & Delete from the cursor to the beginning of line. \\
\hline
\multicolumn{2}{|l|}{\small\it{Undo, Redo and Copy/Paste Commands:}} \\
\hline
\textbf{u} & Undo previous text modification. \\
\hline
\textbf{U} & Undo all previous text modifications on the line. \\
\hline
\textbf{.} & Redo the last text modification. \\
\hline
\textbf{y{\textless}movement command{\textgreater}} & Yank a movement into buffer (copy). \\
\hline
\textbf{yy} & Yank the whole line. \\
\hline
\textbf{p} & Insert the yanked text at the cursor (paste). \\
\hline
\textbf{P} & Insert the yanked text before the cursor. \\
\hline
\multicolumn{2}{|l|}{\small\it{Commands for Command History:}} \\
\hline
\textbf{k} & Move backward one command in history. \\
\hline
\textbf{j} & Move forward one command in history. \\
\hline
\textbf{G} & Move to the history line \textit{n} (you may specify the argument \textit{n} by typing it on number keys, for example, \textbf{15G}). \\
\hline
\textbf{/\textit{string}} or \textbf{CTRL-r} & Search history backward for a command matching \textit{string}. \\
\hline
\textbf{?\textit{string}} or \textbf{CTRL-s} & Search history forward for a command matching \textit{string} (remember that on most machines CTRL-s stops the output to the terminal (CTRL-q starts output) and you might need to change it with stty command). \\
\hline
\textbf{n} & Repeat search in the same direction as previous. \\
\hline
\textbf{N} & Repeat search in the opposite direction as previous. \\
\hline
\multicolumn{2}{|l|}{\small\it{Completion Commands:}} \\
\hline
\textbf{TAB} or \textbf{CTRL-i} or \textbf{=} & List possible completions. \\
\hline
\textbf{*} & Insert all possible completions. \\
\hline
\multicolumn{2}{|l|}{\small\it{Miscellaneous Commands:}} \\
\hline
\textbf{$\sim$} & Invert the case of the character under cursor, and move a character right. \\
\hline
\textbf{\#} & Prepend \textbf{\#} (comment character) to the line and send it to the history list. \\
\hline
\textbf{\_} &  Inserts the \textbf{n}-th word of the previous command in the current line. \\
\hline
\textbf{0, 1, 2, ...} & Sets the numeric argument. \\
\hline
\textbf{CTRL-v} & Insert a character literally (quoted insert). \\
\hline
\textbf{CTRL-t} & Transpose (exchange) two characters. \\
\hline
\end{tabular}

\vspace{2mm}
\begin{center}
\large Examples and Tips
\end{center}

\begin{itemize}
\item Some of the commands take a \textbf{{\textless}movement command{\textgreater}}. These commands apply the movement to themselves. For example, \textbf{d\$} would use \$ as a movement, which moves the cursor to the end of the line, thus, the whole \textbf{d\$} would delete text from the current cursor position to the end of the line. Another example, a command \textbf{c\textit{fA}} would use \textbf{fA} as a movement, which finds the next occurance of the character {\bf A}, thus, the whole command would change the line up to character {\bf A}.
\item Use \textbf{CTRL-v} to insert character literally, for example, \textbf{CTRL-v CTRL-r} would insert CTRL-r in the command line.
\item See \textbf{man bash}, \textbf{man readline}, and built in \textbf{bind} command for modifying the default behavior!
\end{itemize}

\vfill

\framebox{\parbox{5in}{
A cheat sheet by \textbf{Peteris Krumins} (peter@catonmat.net), 2008.

\href{http://www.catonmat.net}{http://www.catonmat.net} - good coders code, great reuse

\vspace{2mm}
\footnotesize{Released under GNU Free Document License.}}}

\end{document}

